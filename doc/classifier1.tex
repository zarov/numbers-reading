\section{Classifieur 1 : profils et classifieur par distance euclidienne minimum}

\subsection{Extraction des profils}
Chaque chiffre récupéré précédemment, il est désormais possible d'extraire les profils gauche et droit pour chacun, sur $d$ composantes. Ainsi, pour chaque chiffre, on utilise l'algorithme suivant :

\begin{figure}[h]
\begin{algorithm}[H]
	découper en 5 lignes avec $linspace$
	
	\For{chaque ligne}{
		\While{pixel\_gauche != noir}{
			profil gauche = pixel\_gauche + 1
		}	
		\While{pixel\_droit != noir}{
			profil droit = pixel\_droit - 1
		}
	}
	normaliser le profil obtenu
\end{algorithm}
\end{figure}

Enfin, on récupère tous ces profils et on les classe en fonction du chiffre auquel ils appartiennent, pour ensuite effectuer la moyenne des profils d'un même chiffre.

\subsection{Apprentissage du classifieur}

\subsection{Décision du classifieur}